%! Author = DIEGO
%! Date = 15/02/2024


\usepackage{comment}

###

\BLOCK{if my_dict is defined and
      my_dict['field_representations'] is defined and
      'X' in my_dict['field_representations'] and
      'postprocessing' in my_dict['field_representations']['X'] and
      my_dict['field_representations']['X']['postprocessing'] is defined and
      my_dict['field_representations']['X']['postprocessing'] is not none}
    % dictionary 'preprocessing' not empty
On th field \verb| X | have been applied the following postprocessing techniques:
\BLOCK{ for key, value in my_dict['field_representations']['X']['postprocessing'].items() }
      \BLOCK{ for k, v in value.items() }
            \verb| \VAR{ k } | ,
      \BLOCK{ endfor }
\BLOCK{ endfor }
\BLOCK{else}
% This block is rendered if no post processing technique has been applied
No postprocessing techniques have been used on the data \lstinline[style=verbatim-text]| X | in this experiment.
\BLOCK{endif}
\hfill\break

% postprocessing_on_ X ___

###

\begin{comment}
Author = DIEGO MICCOLI
Alias = Kozen88
Organization = SWAP Research Group UniBa
Date = 27-12-2023

This mini template is not working by itself because there are latex command missing needed
to compile the file and give as output a pdf file, in addition it has been added jinja
statement in order to control the rendering of the latex file with the jinja library, for these
reasons it needs to be used with the other mini chunks in conjunction.
    BLOCK{ if v is mapping }
       BLOCK{ else }
            \begin{itemize}
                \item no post processing techniques applied on \VAR{ k }
            \end{itemize}
       BLOCK{ endif }
\end{comment}
