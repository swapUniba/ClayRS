%! Author = DIEGO MICCOLI
%! Date = 27/12/2023

\usepackage{comment}


###

\BLOCK{if my_dict['recsys'] is defined and
      my_dict['recsys']['ContentBasedRS'] is defined and
      my_dict['recsys']['ContentBasedRS']['algorithm']['IndexQuery'] is defined}
% INDEX QUERY ALGO
The algorithm used is the Index Query.
\begin{itemize}
    \item item fields: Y
    \item user fields: W
    \item classic similarity: \VAR{my_dict['recsys']['ContentBasedRS']['algorithm']['IndexQuery']['classic_similarity']|default('no used')}
    \item threshold: \VAR{my_dict['recsys']['ContentBasedRS']['algorithm']['IndexQuery']['threshold']|default('no threshold')}
    \item embedding combiner: Z
\end{itemize}
\hfill\break
\hfill\break
The mode used is \VAR{ my_dict['recsys']['ContentBasedRS']['mode']} and the number of recommendation given is
\VAR{ my_dict['recsys']['ContentBasedRS']['n_recs']|default('no number has been setted')}.
The methodology used is H .
\hfill\break
\hfill\break
% index query end____
\BLOCK{endif}

###

\begin{comment}
Author = DIEGO MICCOLI
Alias = Kozen88
Organization = SWAP Research Group UniBa
Date = 27-12-2023

This mini template is not working by itself because there are latex command missing needed
to compile the file and give as output a pdf file, in addition it has been added jinja
statement in order to control the rendering of the latex file with the jinja library, for these
reasons it needs to be used with the other mini chunks in conjunction.
\end{comment}