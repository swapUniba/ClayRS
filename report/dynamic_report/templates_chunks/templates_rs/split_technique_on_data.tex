%! Author = DIEGO
%! Date = 16/02/2024
\usepackage{comment}

###

% ------------------------------ START SUBSECTION OF PARTITIONING OF RECSYS --------------------------------------------
% subsection of the splitting technique used, referred to partition protocol
\subsection{Data splitting technique}\label{subsec:partitioning}
\BLOCK{if my_dict['partitioning'] is defined and
        my_dict['partitioning']['KFoldPartitioning'] is defined}
% KFOLD PARTITIONING TECNIQUE
K-fold cross-validation is a technique used in machine learning to assess the performance of a predictive model.
The basic idea is to divide the dataset into K subsets, or folds.
The model is then trained K times, each time using K-1 folds for training and the remaining fold for validation.
This process is repeated K times, with a different fold used as the validation set in each iteration.
\hfill\break
The KFoldPartitioning has been used with the following setting:
\hfill\break
\BLOCK{if my_dict.get('partitioning', {}).get('KFoldPartitioning', {}).get('shuffle') == True}
The data has been shuffled before being split into batches.
\BLOCK{endif}
The partitioning technique has been executed with the following settings:
\begin{itemize}
    \item number of splits: \VAR{my_dict['partitioning']['KFoldPartitioning']['n_splits']}
    \item shuffle: \VAR{my_dict['partitioning']['KFoldPartitioning']['shuffle']}
    \item random state: \VAR{my_dict['partitioning']['KFoldPartitioning']['random_state']|default('no random state applied')}
    \item skip user error: \VAR{my_dict['partitioning']['KFoldPartitioning']['skip_user_error']|default('no setted')}
\end{itemize}
\hfill\break
% KFOLD PARTITIONING TECNIQUE ended
\BLOCK{endif}

\BLOCK{if my_dict['partitioning'] is defined and
        my_dict['partitioning']['HoldOutPartitioning'] is defined}
%  HOLD-OUT PARTIONING TECNIQUE
The partitioning used is the Hold-Out Partitioning.
This approach splits the dataset in use into a train set and a test set.
The training set is what the model is trained on, and the test set is used to see how
well the model will perform on new, unseen data.
\hfill\break
The train set size of this experiment is the \VAR{my_dict['partitioning']['HoldOutPartitioning']['train_set_size'] * 100}\%
of the original dataset, while the test set is the remaining \VAR{(100 - (my_dict['partitioning']['HoldOutPartitioning']['train_set_size'] * 100))}\%.
\hfill\break
\BLOCK{ if my_dict.get('partitioning', {}).get('HoldOutPartitioning', {}).get('shuffle') == True }
The data has been shuffled before being split into batches.
\BLOCK{endif}
\hfill\break
%  HOLD-OUT PARTIONING TECNIQUE ended
\BLOCK{endif}
% end partitioning section___________
###


\begin{comment}
Author = DIEGO MICCOLI
Alias = Kozen88
Organization = SWAP Research Group UniBa
Date = 27-12-2023

This mini template is not working by itself because there are latex command missing needed
to compile the file and give as output a pdf file, in addition it has been added jinja
statement in order to control the rendering of the latex file with the jinja library, for these
reasons it needs to be used with the other mini chunks in conjunction.
\end{comment}
